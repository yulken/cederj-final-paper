\begin{capitulo}{Conceitos} \label{cap:concepts}


\paragrafo{Este capítulo tem como objetivo apresentar os conceitos provenientes dos livros e que serão utilizados na aplicação Web deste trabalho. }

\begin{secao}{Clean Architecture} \label{sec:carch}

\paragrafo{O livro Arquitetura Limpa, de Robert C. Martin, aborda vários conceitos importantíssimos referentes a arquitetura e facilidade de manutenção de sistemas. Neste trabalho será descrito apenas o que foi de fato implementado no sistema, sendo assim, os Princípios de Design (SOLID) e o próprio conceito de Arquitetura Limpa }

\begin{subsecao}{Arquitetura} \label{subsec:arquitetura}
  \paragrafo{Segundo Robert C. Martin, "Todo sistema de software fornece dois valores diferentes para os stakeholders: comportamento e estrutura. Desenvolvedores de software são responsáveis por garantir que ambos esses valores permaneçam altos."}
  \paragrafo{O comportamento se trata da própria funcionalidade do sistema, ou seja, como o sistema se comporta de forma a economizar ou gerar dinheiro para os stakeholders. Se esse sistema para de funcionar ou necessita de uma manutenção, o desenvolvedor cria mais código a fim de tratar os problemas levantados.}
  \paragrafo{A estrutura, ou como iremos tratar nesse trabalho, a Arquitetura, se trata da forma que essas funcionalidades são desenvolvidas e o quão fácil é mudar esse sistema. Assim, quando os stakeholders decidem mudar o curso do projeto, a implementação dessa mudança não deveria ser mais díficil de ser realizada do que o próprio escopo da mudança.}
  \paragrafo{O problema principal que o autor levanta ao escrever o livro se trata de tanto desenvolvedores quanto gerentes de negócio priorizarem massivamente o primeiro, ignorando parcial ou totalmente o segundo valor que o software tem a oferecer. O autor argumenta, analisando as situações extremas:}
  
  \begin{itemize}[label=\raisebox{0.25ex}{\normalsize$\bullet$}]
    \vspace{-0.3cm}
    \setlength{\itemindent}{2cm}
    \item \hspace{0.1cm} "Se você me der um programa que funcione perfeitamente, mas seja impossível de mudar, então ele não funcionará quando as exigências mudarem e eu não serei capaz de fazê-lo funcionar. Portanto, o programa será inútil."
    \item \hspace{0.1cm} "Se você me der um programa que não funcione, mas seja fácil de mudar, então eu posso fazê-lo funcionar, e posso mantê-lo funcionando à medida que as exigências mudarem. Portanto, o programa permanecerá continuamente útil."
    \vspace{0.6cm}
  \end{itemize}

  \paragrafo{Concluindo este tema, o objetivo de termos uma boa arquitetura de software pode ser descrito, conforme diz o próprio autor, de forma utópica, da seguinte forma: "O objetivo da arquitetura de software é minimizar os recursos humanos necessários para construir e manter um determinado sistema."}

\end{subsecao}

\begin{subsecao}{SOLID} \label{subsec:solid}
  \paragrafo{Os princípios SOLID, em resumo, se tratam de regras de organização de funções e estruturas de dados em agrupamentos. O principal objetivo de sua aplicação está em criar software que seja de fácil entendimento, manutenção e reaproveitamento.}
  \paragrafo{Também proposto por Robert C. Martin, SOLID é um acrônimo para outras siglas:}
  \begin{itemize}[label=\raisebox{0.25ex}{\normalsize$\bullet$}]
    \vspace{-0.3cm}
    \setlength{\itemindent}{2cm}
    \item \hspace{0.1cm} Princípio da Responsabilidade Única (\emph{Single Responsiblity Principle})
    \item \hspace{0.1cm} Princípio do Aberto/Fechado (\emph{Open-Closed Principle})
    \item \hspace{0.1cm} Princípio de Substituição de Liskov (\emph{Liskov Substitution Principle})
    \item \hspace{0.1cm} Princípio da Segregação de Interface (\emph{Interface Segregation Principle})
    \item \hspace{0.1cm} Princípio da Inversão de Dependência (\emph{Dependency Inversion Principle})
    \vspace{0.6cm}
  \end{itemize}
  \paragrafo{Definindo os príncipios, bem resumidamente, teríamos:}
  \begin{itemize}[label=\raisebox{0.25ex}{\normalsize$\bullet$}]
    \vspace{-0.3cm}
    \setlength{\itemindent}{2cm}
    \item \hspace{0.1cm} \acs{SRP} - Cada módulo de software deve ser estruturado de forma que haja uma, e apenas uma, razão para mudar.
    \item \hspace{0.1cm} \acs{OCP} - Para que o software seja fácil de mudar, deve ser projetado para garantir que o mesmo está aberto à adição, mas fechado para alteração.
    \item \hspace{0.1cm} \acs{LSP} - Criando um sistema a partir de partes trocáveis entre si, estas partes devem firmar um contrato que permita que uma seja substituída pela outra.
    \item \hspace{0.1cm} \acs{ISP} - Deve-se evitar que o software dependa de métodos que não usa.
    \item \hspace{0.1cm} \acs{DIP} - Código de nível mais alto (Próximo às regras de negócio) não deve depender de código de nível mais baixo (detalhes de implementação, \emph{frameworks}, banco de dados).
    \vspace{0.6cm}
  \end{itemize}
  \paragrafo{Os exemplos da implementação desses princípios serão fornecidos posteriormente neste mesmo trabalho, no Capítulo 4.}
\end{subsecao}

\begin{subsecao}{Arquitetura Limpa} \label{subsec:carch}
\paragrafo{Robert C. Martin menciona que nas últimas décadas foram propostas várias ideias relacionadas à arquitetura de sistemas. O autor relata que embora as arquiteturas variem entre si à nível de detalhes, elas ainda são muito similares frente à seus objetivos principais: A Separação de Preocupações.}
\paragrafo{A Separação de preocupações (\emph{Separation of Concerns}) consistiria em dividir, em níveis, todo o código desenvolvido. Sendo assim, as arquiteturas tem módulos de lógica de negócios testáveis, independentes de frameworks, UI, banco de dados ou qualquer agência externa.}
\paragrafo{Em resumo, as características citadas acima podem definir uma arquitetura limpa.}
\end{subsecao}
\end{secao}

\begin{secao}{Design Patterns} \label{sec:patterns}

  \paragrafo{Como alguns princípios vistos na seção de Clean Architecture, os Design Patterns tem como principal função garantir a reusabilidade do código com baixo acoplamento. De fato, a proposta principal do livro está em ser um catálogo de soluções para problemas de sistemas orientados a objetos.}
  \paragrafo{No total,são 23 padrões de projetos diferentes, subdividos em 3 tipos: Padrões de Criação, Padrões Comportamentais e Padrões Estruturais:}
  
  \paragrafo{Padrões de Criação}
  \begin{itemize}[label=\raisebox{0.25ex}{\normalsize$\bullet$}]
    \vspace{-0.3cm}
    \setlength{\itemindent}{2cm}
    \item \hspace{0.1cm} Abstract Factory
    \item \hspace{0.1cm} Builder
    \item \hspace{0.1cm} Factory Method
    \item \hspace{0.1cm} Prototype
    \item \hspace{0.1cm} Singleton
    \vspace{0.6cm}
  \end{itemize}

  \paragrafo{Padrões Estruturais}
  \begin{itemize}[label=\raisebox{0.25ex}{\normalsize$\bullet$}]
    \vspace{-0.3cm}
    \setlength{\itemindent}{2cm}
    \item \hspace{0.1cm} Adapter
    \item \hspace{0.1cm} Bridge
    \item \hspace{0.1cm} Composite
    \item \hspace{0.1cm} Decorator
    \item \hspace{0.1cm} Facade
    \item \hspace{0.1cm} Flyweight
    \item \hspace{0.1cm} Proxy
    \vspace{0.6cm}
  \end{itemize}
  
  \paragrafo{Padrões Comportamentais}
  \begin{itemize}[label=\raisebox{0.25ex}{\normalsize$\bullet$}]
    \vspace{-0.3cm}
    \setlength{\itemindent}{2cm}
    \item \hspace{0.1cm} Chain of Responsiblity
    \item \hspace{0.1cm} Command
    \item \hspace{0.1cm} Interpreter
    \item \hspace{0.1cm} Iterator
    \item \hspace{0.1cm} Mediator
    \item \hspace{0.1cm} Memento
    \item \hspace{0.1cm} Observer
    \item \hspace{0.1cm} State
    \item \hspace{0.1cm} Strategy
    \item \hspace{0.1cm} Template Method
    \item \hspace{0.1cm} Visitor
    \vspace{0.6cm}
  \end{itemize}
  
  \paragrafo{Não faz parte da proposta desse trabalho usar cada um dos padrões supracitados. A proposta está em identificar as oportunidades de aplicação durante as etapas de desenvolvimento. Como o sistema, até o atual momento, não está pronto, será necessário revisitar esse capítulo posteriormente listando e explicando cada padrão usado na aplicação web.}
\end{secao}

\end{capitulo}