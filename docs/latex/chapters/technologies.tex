\begin{capitulo}{Tecnologias} \label{cap:technologies}

\paragrafo{Neste capítulo serão abordadas as tecnologias usadas no
desenvolvimento da aplicação web. A linguagem que permeará tanto o
\emph{frontend} quanto o \emph{backend} será o \emph{JavaScript}, porém, como as
práticas abordadas neste trabalho são mais direcionadas ao \emph{backend}, não
será descrito com detalhes as tecnologias do \emph{frontend}.}
 
\begin{secao}{JavaScript e TypeScript} \label{sec:pl} \paragrafo{O JavaScript,
também conhecido como ECMAScript ou JS, é uma linguagem leve e interpretada que
pode ser executada tanto \emph{client-side}, como nos navegadores, quanto
\emph{server-side}, como em servidores remotos.} \paragrafo{Na época de sua
padronização, era esperado que fosse apenas uma simples linguagem de scripts
para navegadores, apenas para usos pontuais em páginas web\cite{typescript}.
Porém, com o tempo, a linguagem se tornou mais popular ao longo de que
desenvolvedores começaram a usá-la com mais frequência para aprimorar a
interacionabilidade de páginas web e hoje a mesma extrapola o ambiente dos
navegadores e roda também nos servidores através do Node.js (Seção
\ref{sec:nodejs}).} \paragrafo{Porém, seu design original não previa o uso em
aplicações muito complexas, e mesmo com as atualizações mais recentes do
ECMAScript, a linguagem pode acabar sendo uma opção insegura para algumas
aplicações empresariais por, dentre alguns motivos, sua fraca tipagem.}
\paragrafo{Por essa desvantagem característica do JavaScript, o
\emph{TypeScript} surge como uma proposta de ser uma camada sobre o JavaScript
que permita checagem estática de tipos. Desta forma, a linguagem garante,
durante o desenvolvimento, alguns recursos que permitam uma maior
previsibilidade em relação ao resultado do código desenvolvido sem alterar o
comportamento de execução do JavaScript.}

\end{secao}
\begin{secao}{Node.js} \label{sec:nodejs} \paragrafo{O \emph{Node.js} é um
    ambiente de execução \emph{JavaScript} assíncrono e orientado a eventos.
    Apesar de ser \emph{single threaded}, o \emph{Node.js} possui o \emph{event
    loop}, que permite o mesmo a executar operações E/S de forma não bloqueante
    ao delegar as mesmas para o kernel do sistema sempre que possível. Ou seja,
    o \emph{event loop} garante que não haja a necessidade de que o código
    \emph{JavaScript} espere o término da execução de algum processo
    não-\emph{JavaScript} para executar a próxima instrução\cite{nodejs}.}
    \paragrafo{Devido à esse diferencial, o \emph{Node.js} pode lidar com um
    número elevado de requisições simultâneas em um único servidor sem a
    necessidade de engessar o projeto com a gerência de threads simultâneas, o
    que simplifica o trabalho do desenvolvedor.} \paragrafo{Uma outra vantagem
    em relação a outros ambientes é que, por aplicações \emph{Node.js} serem
    escritas em \emph{JavaScript}, parte do aprendizado necessário para
    desenvolver para o \emph{frontend} acaba sendo reutilizado no \emph{backend}
    sem a necessidade de aprender uma nova linguagem de programação para o uso
    no servidor.} 
  \end{secao}
\vspace{15mm}
\begin{secao}{Bibliotecas e Frameworks} \label{sec:frameworks} \paragrafo{Nesta
  seção serão descritos os \emph{frameworks} ou bibliotecas mais relevantes para
  o backend desta aplicação. No caso, apenas aqueles que são cruciais para o
  funcionamento da mesma.}
  \begin{subsecao}{TypeORM e PostgreSQL}\label{subsec:typeorm} \paragrafo{O
    Sistema de Gerenciamento de Banco de Dados escolhido para este trabalho foi
    o PostgreSQL. Dentre as razões que motivam essa escolha \cite{postgresql}:}
    \begin{lista}
      \itemlista{Seu uso é gratuito}
      \itemlista{É popular}
      \itemlista{\emph{Open Source} com uma comunidade ativa} 
      \itemlista{É um projeto sólido com mais de 30 anos de desenvolvimento}
      \itemlista{Bastante compatibilidade com \acs{ORM}s das tecnologias
      apresentadas}
     \end{lista}
    \paragrafo{Para manipularmos o acesso ao banco de dados PostgreSQL, iremos
    um utilizar uma ferramenta \emph{Object-Relational Mapping}, ou \acs{ORM}. O
    que motiva esse uso é a facilidade que as ferramentas \acs{ORM}s pode trazer
    ao processo de desenvolvimento ao abstrair completamente a camada de
    comunicação com o Banco de Dados.} \paragrafo{Isto é, o desenvolvedor não
    precisa conhecer muitos detalhes de implementação ou sintaxe de algum
    Sistema de Banco de Dados específico, além de não precisar se preocupar com
    a sanitização das \emph{queries} ou com outros aspectos relacionados às
    chamadas ao banco de dados.} \paragrafo{O \acs{ORM} utilizado nesse trabalho
    será o TypeORM, um \emph{framework} voltado principalmente para o
    TypeScript. A preferência por esse \acs{ORM} se dá por vários motivos.
    Primeiramente, a compatibilidade com os principais bancos de dados
    relacionais, com o MongoDB (em estado experimental) e com as principais
    plataformas onde o JavaScript é suportado (Node.js, Navegador, React Native,
    entre outros).\cite{typeorm_home}} \paragrafo{Mas o motivo principal para a
    escolha desse \acs{ORM} está na sua adoção pelo padrão Data Mapper. Este
    padrão de \acs{ORM} está bastante relacionado com as ideias de facilidade de
    manutenção já abordadas no Capítulo \ref{cap:concepts}, visto que o uso do
    mesmo promove a separação de preocupações e, por consequência, baixo
    acoplamento e maior qualidade de
    código.\cite{typeorm_data_mapper}\cite{typeorm_home}}

  \end{subsecao}
  \begin{subsecao}{Express}\label{subsec:express}
    \notaeditor{Nota: A preencher}
  \end{subsecao}

  \begin{subsecao}{Tsyringe}\label{subsec:tsyringe}
    \notaeditor{Nota: A preencher}
  \end{subsecao}

\end{secao}
\end{capitulo}