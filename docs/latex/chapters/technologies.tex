\begin{capitulo}{Tecnologias} \label{cap:technologies}

\paragrafo{Neste capítulo serão abordadas as tecnologias usadas no desenvolvimento da aplicação web. A linguagem que permeará tanto o \emph{frontend} quanto o \emph{backend} será o \emph{JavaScript}.}
 
\begin{secao}{JavaScript e TypeScript} \label{sec:pl}
\paragrafo{O JavaScript, também conhecido como ECMAScript ou JS, é uma linguagem leve e interpretada que pode ser executada tanto \emph{client-side}, como nos navegadores, quanto \emph{server-side}, como em servidores remotos.}
\paragrafo{Na época de sua padronização, era esperado que fosse apenas uma simples linguagem de scripts para navegadores, apenas para usos pontuais em páginas web\cite{typescript}. Porém, com o tempo, a linguagem se tornou mais popular ao longo de que desenvolvedores começaram a usá-la com mais frequência para aprimorar a interacionabilidade de páginas web e hoje a mesma extrapola o ambiente dos navegadores e roda também nos servidores através do Node.js (Seção \ref{subsec:nodejs}).}
\paragrafo{Porém, seu design original não previa o uso em aplicações muito complexas, e mesmo com as atualizações mais recentes do ECMAScript, a linguagem pode acabar sendo uma opção insegura para algumas aplicações empresariais por, dentre alguns motivos, sua fraca tipagem.}
\paragrafo{Por essa desvantagem característica do JavaScript, o \emph{TypeScript} surge como uma proposta de ser uma camada sobre o JavaScript que permita checagem estática de tipos. Desta forma, a linguagem garante, durante o desenvolvimento, alguns recursos que permitam uma maior previsibilidade em relação ao resultado do código desenvolvido sem alterar o comportamento de execução do JavaScript.}

\end{secao}
\begin{secao}{Backend} \label{sec:backend}
  \begin{subsecao}{Node.js}\label{subsec:nodejs}
    \paragrafo{O \emph{Node.js} é um ambiente de execução \emph{JavaScript} assíncrono e orientado a eventos. Apesar de ser \emph{single threaded}, o \emph{Node.js} possui o \emph{event loop}, que permite o mesmo a executar operações E/S de forma não bloqueante ao delegar as mesmas para o kernel do sistema sempre que possível. Ou seja, o \emph{event loop} garante que não haja a necessidade de que o código \emph{JavaScript} espere o término da execução de algum processo não-\emph{JavaScript} para executar a próxima instrução\cite{nodejs}.}
    \paragrafo{Devido à esse diferencial, o \emph{Node.js} pode lidar com um número elevado de requisições simultâneas em um único servidor sem a necessidade de engessar o projeto com a gerência de threads simultâneas, o que simplifica o trabalho do desenvolvedor.}
    \paragrafo{Uma outra vantagem em relação a outros ambientes é que, por aplicações \emph{Node.js} serem escritas em \emph{JavaScript}, parte do aprendizado necessário para desenvolver para o \emph{frontend} acaba sendo reutilizado no \emph{backend} sem a necessidade de aprender uma nova linguagem de programação para o uso no servidor.} 
  \end{subsecao}
\end{secao}
\end{capitulo}