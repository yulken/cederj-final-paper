
\paragrafo{No fluxo descrito na seção anterior já é possível encontrar alguns
conceitos SOLID sendo implementados na aplicação web. Porém, será necessário ver
outros fluxos para obter exemplos de implementação que sigam todos os
princípios.}

\begin{subsecao}{Implementação e Benefícios do \acs{SRP}}\label{subsec:dev_srp}

\paragrafo{O Princípio da Responsabilidade Única (\acs{SRP}) está sendo seguido
na classe ShowGameService. Isso ocorre pois a classe tem apenas um propósito,
que é a exibição dos detalhes do jogo. Sendo assim, a mesma tem um escopo mais
restrito em razões para se necessitar de alteração, por exemplo, adicionar
validações, passar a buscar os jogos pelo nome em vez do id, entre outros.}

\vspace{5mm}
\begin{lstlisting}[language=JavaScript, caption={ShowGameService.ts},captionpos=b, label=alg:showgameservice]
import AppError from '@shared/errors/AppError';
import log from '@shared/utils/log';
import { injectable, inject } from 'tsyringe';
import Game from '../infra/typeorm/entities/Game';

import IGamesRepository from '../repositories/IGamesRepository';

interface IRequest {
  game_id: string;
}

@injectable()
export default class ShowGameService {
  constructor(
    @inject('GamesRepository')
    private gamesRepository: IGamesRepository,
  ) {}

  public async execute({ game_id }: IRequest): Promise<Game> {
    log.debug(`Show Game :: id: ${game_id}`);
    const game = await this.gamesRepository.findById(game_id);
    if (!game) {
      throw new AppError('Game not found', 404);
    }
    return game;
  }
}
\end{lstlisting}
\vspace{5mm}
  
\paragrafo{Uma forma fácil de imaginar a classe acima ferindo este princípio
seria se a mesma além de exibir jogos, também criasse ou atualizasse os mesmos.
Com isso, a manutenção dessa classe pode ser mais difícil de se realizar do que
a da primeira:}

\vspace{5mm}
\begin{lstlisting}[language=JavaScript, caption={Implementação que viola o SRP},captionpos=b, label=alg:gameservice]
@injectable()
export default class GameService {
  constructor(
    @inject('GamesRepository')
    private gamesRepository: IGamesRepository,
  ) {}

  public async showGame({ game_id }: IShowGameDTO): Promise<Game> {
    //validacoes e processos para exibir detalhes de um jogo
  }

  public async createGame({
      name,
      price,
      publisher,
      release_date,
    }: ICreateGameDTO): Promise<Game> {

    //validacoes e processos para criar um jogo
  }

  public async updateGame({
    game_id,
    price,
    release_date,
  }: IUpdateGameDTO): Promise<Game> {
    const game = await this.gamesRepository.findById(game_id);

    //validacoes e processos para atualizar um jogo
  }
}
\end{lstlisting}
\vspace{5mm}

\paragrafo{Ao absorver muitas responsabilidades, a classe GameService passa a
englobar diversos comportamentos e, com isso, passa a exigir mais esforço para
se manusear.}

\paragrafo{Desta forma, o leitor pode se questionar porque a classe
 GamesController não viola o \acs{SRP}, visto que a mesma realiza todas essas
 operações. Isso não se caracteriza como uma violação pois a responsabilidade da
 classe é apenas comunicar as chamadas recebidas via HTTP para a aplicação. O
 \emph{Controller} desconhece os fluxos de exibição, criação ou atualização, mas
 conhece as classes que contêm esses fluxos.}
\end{subsecao}
\begin{subsecao}{Implementação e Benefícios do \acs{OCP}}\label{subsec:dev_ocp}

\paragrafo{Para exemplificar o Princípio Aberto Fechado (\acs{OCP}), usaremos o
fluxo responsável pela reinvidicação de cartões pré-pagos. Para usar esse
serviço, o usuário insere o código de cartão um cartão pré-pago e resgata jogos
ou dinheiro para a compra de jogos na loja.} 

\paragrafo{Porém, ao cadastrar esses códigos no sistema, a validação para esses
dois tipos de items a ser reinvidicados é diferente. Os cartões de dinheiro
devem resgatar somente os valores de R\$30, R\$50 ou R\$100. Para validar um
jogo, somente é necessário que este exista.}

\paragrafo{Com os requisitos acima, um desenvolvedor pode pensar em criar a 
seguinte classe:}

\vspace{5mm}
\begin{lstlisting}[language=JavaScript, caption={CreateCodeService.ts},captionpos=b, label=alg:createcodeservice]
import { injectable, inject } from 'tsyringe';
import { v4 as uuidv4 } from 'uuid';

import AppError from '@shared/errors/AppError';
import IGamesRepository from '@modules/games/repositories/IGamesRepository';
import IGamestoreCodesRepository from '../repositories/IGamestoreCodeRepository';
import GamestoreCode from '../infra/typeorm/entities/GamestoreCode';

interface IRequest {
  cash?: number;
  game_id?: string;
}

@injectable()
export default class CreateCodeService {
  constructor(
    @inject('GamestoreCodesRepository')
    private gamestoreCodesRepository: IGamestoreCodesRepository,

    @inject('GamesRepository')
    private gamesRepository: IGamesRepository,
  ) {}

  public async execute(request: IRequest): Promise<GamestoreCode | undefined> {
    let code;
    const { cash, game_id } = request;

    if (!cash && !game_id) {
      throw new AppError('Internal Server Error', 500);
    }

    if (cash) {
      if (cash !== 30 && cash !== 50 && cash !== 100) {
        throw new AppError('Invalid cash quantity');
      }

      code = await this.gamestoreCodesRepository.create({
        code: uuidv4(),
        is_redeemed: false,
        product: {
          cash,
        },
      } as GamestoreCode);
    }

    if (game_id) {
      if (!(await this.gamesRepository.findById(game_id))) {
        throw new AppError('Game does not exist');
      }

      code = await this.gamestoreCodesRepository.create({
        code: uuidv4(),
        is_redeemed: false,
        product: {
          game: game_id,
        },
      } as GamestoreCode);
    }
    return code;
  }
}

\end{lstlisting}
\vspace{5mm}

\paragrafo{Embora atenda a demanda, essa estratégia fere o \acs{OCP}. Note que,
se em algum momento no futuro for necessário reinvidicar um terceiro item,
teremos que adicionar mais um if ao método principal dessa classe, o que deixará
o código ainda mais acoplado.} 

\paragrafo{Para ilustrar isso, suponhamos que será necessário reinvidicar
códigos que resgatem Pacotes de Expansão para jogos já cadastrados em nossa
plataforma. Essa funcionalidade extrapola o escopo desse trabalho, mas sua
implementação nessa classe seria algo como o exemplo a seguir:}

\vspace{5mm}
\begin{lstlisting}[language=JavaScript, caption={Implementação que viola o OCP},captionpos=b, label=alg:ocpviolation]
interface IRequest {
  cash?: number;
  game_id?: string;
  dlc_id?: string;
}

@injectable()
export default class CreateCodeService {
  constructor(
    @inject('GamestoreCodesRepository')
    private gamestoreCodesRepository: IGamestoreCodesRepository,

    @inject('GamesRepository')
    private gamesRepository: IGamesRepository,

    @inject('DlcsRepository')
    private dlcsRepository: IDlcsRepository,
  ) {}

  public async execute(request: IRequest): Promise<GamestoreCode | undefined> {
    if (!request.cash && !request.game_id && !request.dlc_id) {
      throw new AppError('Internal Server Error', 500);
    }

    if (request.cash) {
      //validacao de cash (R$30, R$50 ou R$100)
      //cadastro do cartao pre pago de cash
    }

    if (request.game_id) {
      //validacao de game (se o game existe)
      //cadastro do cartao pre pago de gane
    }

    if (request.dlc_id) {
      //validacao de dlc (se a dlc existe)
      //cadastro do cartao pre pago de gane
    }
  }
}
\end{lstlisting}
\vspace{5mm}

\paragrafo{Para não ferir este princípio, basta estendermos a funcionalidade de
uma classe base e realizar as validações nas classes herdeiras. A seguir temos
os passos para realizar esse procedimento:} 

\paragrafo{Primeiramente, definimos uma classe abstrata que concentrará os
métodos que serão realizados pelas classes filhas desta. Como os fluxos de
cadastro de códigos para jogo e dinheiro virtual passam por processos diferentes
de validação e cadastro, este serão métodos abstratos desta classe.}

\vspace{5mm}
\begin{lstlisting}[language=JavaScript, caption={Classe Abstrata},captionpos=b, label=alg:abstractcode]
  export default abstract class AbstractCodeTemplate {
    public async execute(request: IRequest): Promise<GamestoreCode> {
      await this.validateData(request);
      const code = await this.createCode(request);
      return code;
    }
  
    protected abstract createCode(request: IRequest): Promise<GamestoreCode>;
    protected abstract validateData(request: IRequest): Promise<void>;
  }
\end{lstlisting}  
\vspace{5mm}

\paragrafo{Em seguida, estendemos, a partir da classe abstrata, uma classe para
cada comportamento diferente. Sendo assim, uma classe para validar e registrar
jogos e outra para dinheiro virtual.}

\vspace{5mm}
\begin{lstlisting}[language=JavaScript, caption={Cadastro de Códigos para Jogos},captionpos=b, label=alg:ocpexample]
interface IRequest {
  game_id: string;
}

@injectable()
export default class CreateGameCodeService extends AbstractCodeTemplate {
  constructor(
    @inject('GamesRepository')
    private gamesRepository: IGamesRepository,

    @inject('GamestoreCodesRepository')
    private gamestoreCodesRepository: IGamestoreCodesRepository,
  ) {
    super();
  }

  protected async createCode({ game_id }: IRequest): Promise<GamestoreCode> {
    //cria codigo para o jogo
  }

  protected async validateData({ game_id }: IRequest): Promise<void> {
    if (!validateUuid(game_id) || !(await this.gamesRepository.findById(game_id))) {
      throw new AppError('Game does not exist');
    }
  }
}
\end{lstlisting}
\vspace{5mm}

\vspace{5mm}
\begin{lstlisting}[language=JavaScript, caption={Cadastro de Códigos para Dinheiro Virtual},captionpos=b, label=alg:cashcodeservice]
@injectable()
export default class CreateCashCodeService extends AbstractCodeTemplate {
  constructor(
    @inject('GamestoreCodesRepository')
    private gamestoreCodesRepository: IGamestoreCodesRepository,
  ) {
    super();
  }

  protected async createCode(data: IRequest): Promise<GamestoreCode> {
    const code = await this.gamestoreCodesRepository.create(/*
      dados do dinheiro virtual
    */);
    return code;
  }

  protected async validateData(data: IRequest): Promise<void> {
    //algoritmo de validacao do dinheiro virtual
  }
}
\end{lstlisting}
\vspace{5mm}

\end{subsecao}
\begin{subsecao}{Implementação e Benefícios do \acs{LSP}}\label{subsec:dev_lsp}

\paragrafo{Um exemplo muito recorrente dos dois pŕoximos princípios, isto é, o
Princípio da Substituição de Liskov (\acs{LSP}) e o Princípio da Segregação de
Interfaces (\acs{ISP}), está no funcionamento das classes \emph{Repository} da
aplicação.} 

\paragrafo{Tomando por exemplo o \emph{Repository} de Usuários,
temos a interface IUsersRepository e suas implementações UsersRepository e
FakeUsersRepository.}

\vspace{5mm}
\begin{lstlisting}[language=JavaScript, caption={Interace a ser implementada},captionpos=b, label=alg:iusersrepository]
import ICreateUserDTO from '../dtos/ICreateUserDTO';
import User from '../infra/typeorm/entities/User';

export default interface IUsersRepository {
  findById(id: string): Promise<User | undefined>;
  findByEmail(email: string): Promise<User | undefined>;
  create(data: ICreateUserDTO): Promise<User>;
  save(user: User): Promise<User>;
}
\end{lstlisting}
\vspace{5mm}

\vspace{5mm}
\begin{lstlisting}[language=JavaScript, caption={UsersRepository.ts},captionpos=b, label=alg:usersrepository]
import { getRepository, Repository } from 'typeorm';
import IUsersRepository from '@modules/users/repositories/IUsersRepository';

import ICreateUserDTO from '@modules/users/dtos/ICreateUserDTO';
import log from '@shared/utils/log';
import User from '../entities/User';

class UsersRepository implements IUsersRepository {
  private ormRepository: Repository<User>;

  constructor() {
    this.ormRepository = getRepository(User);
  }

  public async findById(id: string): Promise<User | undefined> {
    log.debug('Users :: findById');
    return this.ormRepository.findOne(id);
  }

  public async findByEmail(email: string): Promise<User | undefined> {
    log.debug('Users :: findByEmail');
    const user = await this.ormRepository.findOne({
      where: { email },
    });
    return user;
  }

  public async create(userData: ICreateUserDTO): Promise<User> {
    log.debug('Users :: create');
    const user = this.ormRepository.create(userData);
    await this.ormRepository.save(user);
    return user;
  }

  public async save(user: User): Promise<User> {
    log.debug('Users :: save');
    return this.ormRepository.save(user);
  }
}

export default UsersRepository;
\end{lstlisting}
\vspace{5mm}

\vspace{5mm}
\begin{lstlisting}[language=JavaScript, caption={FakeUsersRepository.ts},captionpos=b, label=alg:fakeusersrepository]
import IUsersRepository from '@modules/users/repositories/IUsersRepository';
import ICreateUserDTO from '@modules/users/dtos/ICreateUserDTO';

import { v4 as uuidv4 } from 'uuid';
import User from '../../../../modules/users/infra/typeorm/entities/User';

export default class FakeUsersRepository implements IUsersRepository {
  private users: User[] = [];

  public async findById(id: string): Promise<User | undefined> {
    return this.users.find(user => user.id === id);
  }

  public async findByEmail(email: string): Promise<User | undefined> {
    return this.users.find(user => user.email === email);
  }

  public async create(userData: ICreateUserDTO): Promise<User> {
    const user = new User();
    Object.assign(user, { id: uuidv4() }, userData);
    this.users.push(user);
    return user;
  }

  public async save(user: User): Promise<User> {
    const findIndex = this.users.findIndex(findUser => findUser.id === user.id);
    this.users[findIndex] = user;
    return user;
  }
}
\end{lstlisting}
\vspace{5mm}

\paragrafo{Podemos dizer que o \acs{LSP} é respeitado na relação entre esses
três módulos pois qualquer uma das implementações da interface IUsersRepository
é válida de ser importada no módulo CreateUserService, visto que nenhuma delas
quebraria a aplicação.} 

\paragrafo{O mesmo não seria verdade se, por exemplo, a
interface não especificasse um método de criação de usuário. Na hora que esse
método fosse invocado no Caso de Uso, a aplicação retornaria um erro.}

\vspace{5mm}
\begin{lstlisting}[language=JavaScript, caption={Violando o LSP},captionpos=b, label=alg:lspviolation]

  // Modulo IUsersRepository
  export default interface IUsersRepository {
    findById(id: string): Promise<User | undefined>;
    findByEmail(email: string): Promise<User | undefined>;
    save(user: User): Promise<User>;
    // metodo create removido
  }

  // Modulo UsersRepository
  class UsersRepository implements IUsersRepository {
  private ormRepository: Repository<User>;

  constructor() {
    this.ormRepository = getRepository(User);
  }

  public async create(userData: ICreateUserDTO): Promise<User> {
    const user = this.ormRepository.create(userData);
    await this.ormRepository.save(user);
    return user;
  }

  //implementacao dos outros metodos da interface
}

//Modulo CreateUserService
export default class CreateUserService {
  constructor(
    @inject('UsersRepository')
    private usersRepository: IUsersRepository,

    @inject('HashProvider')
    private hashProvider: IHashProvider,
  ) {}

  public async execute(data: IRequest): Promise<User> {
    // validacao dos dados

    // linha abaixo tem erro
    const user = await this.usersRepository.create(/* dados para criacao do usuario */);

    await this.usersRepository.save(user);

    return user;
  }
}

  \end{lstlisting}
\vspace{5mm}

\notaeditor{Nota: Completar a seção com o ISP e DIP}

\end{subsecao}