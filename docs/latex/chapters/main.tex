\begin{capitulo}{Usando os conceitos em uma aplicação Web} \label{cap:main}

\paragrafo{O propósito desse capítulo é demonstrar como os conceitos explicitados 
no Capítulo \ref{cap:concepts} podem ser utilizados na construção de uma aplicação web.}
\begin{secao}{Desenvolvendo a Arquitetura Limpa} \label{sec:dev_carch}

\paragrafo{Na Seção \ref{subsec:clean-architecture} foi explicado a Regra da 
Dependência e o diagrama da Arquitetura Limpa (Figura \ref{fig:carch}). Esses 
príncipios foram mantidos na aplicação desenvolvida para este trabalho e, para 
exemplificar o uso desses conceitos, tomamos como exemplo o fluxo de inserção de novos jogos na plataforma.}
\begin{subsecao}{A implementação}\label{subsec:dev_carch_implementation}
\paragrafo{Primeiramente, temos as Regras de Negócio de Empresa, ou, como implementado na aplicação, as 
Entidades. No contexto delineado, a Entidade abordada é a classe Game.
Assim como descrito na Seção \ref{subsec:clean-architecture}, a Entidade Game, como um 
componente de alto nível, não deve depende de nenhum outro ponto da apĺicação web.}

\vspace{10mm}
\begin{lstlisting}[language=JavaScript, caption={A Entidade Game.ts},captionpos=b, label=alg:game]
import {
    Entity,
    Column,
    PrimaryGeneratedColumn,
    CreateDateColumn,
    UpdateDateColumn,
    } from 'typeorm';
    
    @Entity('games')
    export default class Game {
    @PrimaryGeneratedColumn('uuid')
    id: string;
    
    @Column()
    name: string;
    
    @Column()
    price: number;
    
    @Column()
    publisher: string;
    
    @Column()
    release_date: Date;
    
    @CreateDateColumn()
    created_at: Date;
    
    @UpdateDateColumn()
    updated_at: Date;
    }
\end{lstlisting}

\vspace{5mm}

\paragrafo{Já a classe que será o Caso de Uso, ou a Regra de Negócio da Aplicação, é a CreateGameService. 
Nota-se que essa classe possui como dependência a entidade Game, assim como previsto na Regra da Dependência.}

\vspace{10mm}
\begin{lstlisting}[language=JavaScript, caption={O Caso de Uso},captionpos=b, label=alg:usecase]
  @injectable()
  export default class CreateGameService {
    constructor(
      @inject('GamesRepository')
      private gamesRepository: IGamesRepository,
    ) {}
  
    public async execute(data: ICreateGameDTO): Promise<Game> {
      //validacoes
      return await this.gamesRepository.create(/* dados do jogo */);
    }
  }
\end{lstlisting}

\vspace{5mm}

\paragrafo{A princípio,é plausível assumir que esta classe estaria violando a Regra da Dependência ao importar o 
módulo IGamesRepository, visto que \emph{Repositories} são módulos de nível inferior aos Casos de Uso. Porém, ao 
inspencionar o conteúdo deste módulo, verifica-se que o mesmo se trata de uma interface.}

\vspace{10mm}
\begin{lstlisting}[language=JavaScript, caption={IGamesRepository},captionpos=b, label=alg:igamerepository]
import ICreateGameDTO from '../dtos/ICreateGameDTO';
import Game from '../infra/typeorm/entities/Game';

export default interface IGamesRepository {
  index(): Promise<Game[]>;
  findById(id: string): Promise<Game | undefined>;
  findByNameAndPublisherAndReleaseDate(
    data: ICreateGameDTO,
  ): Promise<Game | undefined>;
  create(data: ICreateGameDTO): Promise<Game>;
  save(game: Game): Promise<Game>;
}
  
\end{lstlisting}
\vspace{5mm}

\paragrafo{Por fim, temos uma classe que implementa a interface acima. O módulo GamesRepository será uma das 
classes responsáveis por se conectar com o \emph{TypeORM}, um dos Frameworks ou Drivers de nossa applicação. Por isso, 
essa classe se encaixa no arquétipo de Adaptadores de Interface.}

\vspace{10mm}
\begin{lstlisting}[language=JavaScript, caption={O Adaptador de Interface do Banco de Dados},captionpos=b, label=alg:gamerepository]
export default class GamesRepository implements IGamesRepository {
  private ormRepository: Repository<Game>;

  constructor() {
    this.ormRepository = getRepository(Game);
  }

  public async findByNameAndPublisherAndReleaseDate({
    name,
    publisher,
    release_date,
  }: ICreateGameDTO): Promise<Game | undefined> {
    return this.ormRepository.findOne({
      where: { name, publisher, release_date: new Date(release_date) },
    });
  }

  public async create(gameData: ICreateGameDTO): Promise<Game> {
    const game = this.ormRepository.create(gameData);
    await this.ormRepository.save(game);
    return game;
  }
}

\end{lstlisting}
\vspace{5mm}

\paragrafo{Até aqui, todos os níveis apresentados na Figura \ref{fig:carch} já foram representados através das 
classes mencionadas nesta seção, do nível mais alto da aplicação até a dependência da camada de banco de dados. 
O mesmo será feito para a dependência da camada web.}
\paragrafo{A classe GamesController tem uma dependência explícita do módulo CreateGameService. Além disso, também 
depende de outros casos de uso da aplicação, como a atualização de dados de jogos (UpdateGameService) e a listagem
ou exibição de detalhes dos mesmos (IndexGameService e ShowGameService).}
\vspace{5mm}
\begin{lstlisting}[language=JavaScript, caption={GamesController.ts},captionpos=b, label=alg:gamescontroller]    
    export default class GamesController {
      public async index(request: Request, response: Response): Promise<Response> {
        const indexGame = container.resolve(IndexGameService);
        const games = await indexGame.execute();
    
        return response.json(games);
      }
    
      public async create(request: Request, response: Response): Promise<Response> {
        const { name, price, publisher, release_date } = request.body;
        const createGame = container.resolve(CreateGameService);
        const game = await createGame.execute({
          name,
          price,
          publisher,
          release_date,
        });
    
        return response.json(classToClass(game));
      }
    }    
\end{lstlisting}

\vspace{5mm}

\paragrafo{Assim como a classe GamesRepository, a GamesController é um tipo de Adaptador de Interface que se 
comunica diretamente com o \emph{Express JS}, nosso framework que lida com as requisições HTTP.}
\end{subsecao}
\begin{subsecao}{Os benefícios}\label{subsec:dev_carch_benefits}
\paragrafo{Os benefícios de adotar esse método de desenvolvimento se torna óbvio ao necessitar reutilizar o caso 
de uso em algum outro fluxo da aplicação. Neste caso, supondo que haja a necessidade de cadastrar grandes volumes 
de jogos na nossa plataforma, seria mais conveniente passar toda essa informação através de um arquivo do que 
inseri-la manualmente através de requisições Web. Por conta disso, foi criado uma nova classe, 
GamesCreateBatch, que processa um arquivo csv e invoca o serviço de criação de games para cada linha.}
\vspace{5mm}
\begin{lstlisting}[language=JavaScript, caption={Caso de Uso reaproveitado para processamento de arquivos},captionpos=b, label=alg:gamescreatebatch]
export default class GamesCreateBatch implements IBatchOperation {
  public async exec(): Promise<void> {
    if (!process.argv[3]) {
      log.error('Filename not provided');
      return;
    }
    const csvPath = process.argv[3];
    let fileContent;

    try {
      fileContent = fs.readFileSync(csvPath);
    } catch (error) {
      log.error('File Not Found');
      return;
    }
    const lines = fileContent.toString().split('\n');

    const createGame = container.resolve(CreateGameService);

    await Promise.all(lines
      .filter(line => line.length !== 0)
      .map(async (line: string): Promise<Game | boolean> => {
        const [name, price, publisher, raw_release_date] = line.split(',');
        try {
          const release_date = new Date(raw_release_date);

          if (!release_date) {
            throw Error('Invalid Line');
          }

          return await createGame.execute({
            name,
            price: Number(price),
            publisher,
            release_date,
          });
        } catch (error) {
          return log.error(error);
        }
      }));
  }
}
\end{lstlisting}
\vspace{5mm}

\paragrafo{Como os níveis desse fluxo foram muito bem segregados ao longo dos módulos, não foi necessário alterar 
nenhuma linha de código das classes responsáveis por conhecer as regras de negócios da aplicação e das responsáveis 
por interagir com o banco de dados.}
\paragrafo{Outro ponto é que se em algum momento for necessário trocar o \emph{Express JS}, somente a classe 
GamesControllerBatch deverá sofrer uma manutenção, o restante, pelo menos referente a esse fluxo, continuará 
intacto.}
\paragrafo{Outro ganho da adoção desse método está na facilidade em testar os componentes dessa aplicação. Por 
exemplo, o módulo CreateGameService.spec.ts testa, com simplicidade, 100\% das linhas do Caso de Uso abordado.}

\vspace{5mm}
\begin{lstlisting}[language=JavaScript, caption={Módulo de teste },captionpos=b, label=alg:creategametest]
describe('CreateGame', () => {
  beforeEach(() => {
    fakeGamesRepository = new FakeGamesRepository();
    //FakeRepository passado como dependencia \/ na instanciacao 
    //do Caso de Uso
    createGame = new CreateGameService(fakeGamesRepository);
  });
  it('should be able to create a new game', async () => {
    const game = await createGame.execute({
      name: 'Cyberpunk',
      price: 80.0,
      publisher: 'CD Projekt Red',
      release_date: new Date(2021, 1, 1),
    });

    expect(game).toHaveProperty('id');
  });
  it('should not be able to create a duplicated game', async () => {
    const releaseDate = new Date(2021, 1, 1);
    await fakeGamesRepository.create({
      name: 'Cyberpunk',
      price: 80.0,
      publisher: 'CD Projekt Red',
      release_date: releaseDate,
    });
    await expect(
      createGame.execute({
        name: 'Cyberpunk',
        price: 80.0,
        publisher: 'CD Projekt Red',
        release_date: releaseDate,
      }),
    ).rejects.toBeInstanceOf(AppError);
  });
});
\end{lstlisting}
\vspace{5mm}

\paragrafo{Para auxiliar nesse propósito, foi criado a classe FakeGamesRepository, cujo propósito está em simular o 
acesso ao banco de dados. Como essa classe é uma implementação da interface IGamesRepository, também não há necessidade de realizar nenhuma alteração no Caso de Uso nem para a realização dos testes e nem caso haja a necessidade de trocar o \emph{TypeORM} por algum outro framework de acesso a banco de dados.}

\vspace{5mm}
\begin{lstlisting}[language=JavaScript, caption={Outra implementação do IGamesRepository},captionpos=b, label=alg:fakegamesrepository]
export default class FakeGamesRepository implements IGamesRepository {
  private games: Game[] = [];

  public async findByNameAndPublisherAndReleaseDate({
    name,
    publisher,
    release_date,
  }: ICreateGameDTO): Promise<Game | undefined> {
    return this.games.find(
      game => game.name === name && game.publisher === publisher,
    );
  }

  public async create(gameData: ICreateGameDTO): Promise<Game> {
    const game = new Game();
    Object.assign(game, { id: uuidv4() }, gameData);
    this.games.push(game);
    return game;
  }
\end{lstlisting}
\vspace{5mm}

\end{subsecao}
\end{secao}
\begin{secao}{Desenvolvendo uma aplicação SOLID} \label{sec:dev_solid}

\paragrafo{No fluxo descrito na seção anterior já é possível encontrar alguns conceitos SOLID sendo implementados
na aplicação web. Mesmo assim, o SOLID na loja de jogos será apresentado em um fluxo diferente: a exibição de detalhes de um jogo.}

\begin{subsecao}{A implementação}\label{subsec:dev_solid_implementation}

\vspace{5mm}
\begin{lstlisting}[language=JavaScript, caption={Implementação correta do SRP},captionpos=b, label=alg:showgameservice]
interface IRequest {
  game_id: string;
}

@injectable()
export default class ShowGameService {
  constructor(
    @inject('GamesRepository')
    private gamesRepository: IGamesRepository,
  ) {}

  public async execute({ game_id }: IRequest): Promise<Game> {
    const game = await this.gamesRepository.findById(game_id);
    if (!game) {
      throw new AppError('Game not found', 404);
    }
    return game;
  }
}
\end{lstlisting}
\vspace{5mm}
  
\paragrafo{}
\end{subsecao}
\end{secao}
\end{capitulo}