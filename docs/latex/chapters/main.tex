\begin{capitulo}{Usando os conceitos em uma aplicação Web} \label{cap:main}

\paragrafo{O propósito desse capítulo é demonstrar como os conceitos
explicitados no Capítulo \ref{cap:concepts} podem ser utilizados na construção
de uma aplicação web.}
\begin{secao}{Desenvolvendo a Arquitetura Limpa} \label{sec:dev_carch}

\paragrafo{Na Seção \ref{subsec:clean-architecture} foi explicado a Regra da
Dependência e o diagrama da Arquitetura Limpa (Figura \ref{fig:carch}). Esses
príncipios foram mantidos na aplicação desenvolvida para este trabalho e, para
exemplificar o uso desses conceitos, tomamos como exemplo o fluxo de inserção de
novos jogos na plataforma}

\begin{subsecao}{A implementação}\label{subsec:dev_carch_implementation}
\paragrafo{Primeiramente, temos as Regras de Negócio de Empresa, ou, como
implementado na aplicação, as Entidades. No contexto delineado, a Entidade
abordada é a classe Game. Assim como descrito na Seção
\ref{subsec:clean-architecture}, a Entidade Game, como um componente de alto
nível, não deve depender de nenhum outro ponto da aplicação web.}

\vspace{10mm}
\begin{lstlisting}[language=JavaScript, caption={A Entidade Game},captionpos=b, label=alg:entity]
    @Entity('games')
    export default class Game {
    @PrimaryGeneratedColumn('uuid')
    id: string;
    
    @Column()
    name: string;
    
    @Column()
    price: number;
    
    @Column()
    publisher: string;
    
    @Column()
    release_date: Date;
    
    @CreateDateColumn()
    created_at: Date;
    
    @UpdateDateColumn()
    updated_at: Date;
    }
\end{lstlisting}

\vspace{5mm}

\paragrafo{Já a classe que será o Caso de Uso, ou a Regra de Negócio da
Aplicação, é a CreateGameService. Nota-se que essa classe possui como
dependência a entidade Game, assim como previsto na Regra da Dependência.}

\vspace{10mm}
\begin{lstlisting}[language=JavaScript, caption={CreateGameService.ts},captionpos=b, label=alg:creategameservice]
  import { injectable, inject } from 'tsyringe';

  import AppError from '@shared/errors/AppError';
  import log from '@shared/utils/log';
  import Game from '../infra/typeorm/entities/Game';
  import IGamesRepository from '../repositories/IGamesRepository';
  import ICreateGameDTO from '../dtos/ICreateGameDTO';
  
  @injectable()
  export default class CreateGameService {
    constructor(
      @inject('GamesRepository')
      private gamesRepository: IGamesRepository,
    ) {}
  
    public async execute({
      name,
      price,
      publisher,
      release_date,
    }: ICreateGameDTO): Promise<Game> {
      log.debug(
        'Create Game :: ',
        JSON.stringify({ name, price, publisher, release_date }),
      );
      const checkGame =
        await this.gamesRepository.findByNameAndPublisherAndReleaseDate({
          name,
          publisher,
          release_date,
        });
      if (checkGame) {
        throw new AppError('Game Already registered');
      }
      const game = await this.gamesRepository.create({
        name,
        price,
        publisher,
        release_date,
      });
  
      return game;
    }
  }
  
  \end{lstlisting}
  
\vspace{5mm}

\paragrafo{A princípio, é plausível assumir que esta classe estaria violando a
Regra da Dependência ao importar o módulo IGamesRepository, visto que
\emph{Repositories} são módulos de nível inferior aos Casos de Uso. Porém, ao
inspencionar o conteúdo deste módulo, verifica-se que o mesmo se trata de uma
interface.}

\vspace{10mm}
\begin{lstlisting}[language=JavaScript, caption={IGamesRepository.ts},captionpos=b, label=alg:igamerepository]
import ICreateGameDTO from '../dtos/ICreateGameDTO';
import Game from '../infra/typeorm/entities/Game';

export default interface IGamesRepository {
  index(): Promise<Game[]>;
  findById(id: string): Promise<Game | undefined>;
  findByNameAndPublisherAndReleaseDate(
    data: ICreateGameDTO,
  ): Promise<Game | undefined>;
  create(data: ICreateGameDTO): Promise<Game>;
  save(game: Game): Promise<Game>;
}
  
\end{lstlisting}
\vspace{5mm}

\paragrafo{Por fim, temos uma classe que implementa a interface acima. O módulo
GamesRepository será uma das classes responsáveis por se conectar com o
\emph{TypeORM}, um dos Frameworks ou Drivers de nossa aplicação. Por isso, essa
classe se encaixa no arquétipo de Adaptadores de Interface.}

\vspace{10mm}
\begin{lstlisting}[language=JavaScript, caption={GamesRepository.ts},captionpos=b, label=alg:gamerepository]
import { getRepository, Repository } from 'typeorm';
import ICreateGameDTO from '@modules/games/dtos/ICreateGameDTO';
import IGamesRepository from '@modules/games/repositories/IGamesRepository';

import log from '@shared/utils/log';
import Game from '../entities/Game';

class GamesRepository implements IGamesRepository {
  private ormRepository: Repository<Game>;

  constructor() {
    this.ormRepository = getRepository(Game);
  }

  public async findByNameAndPublisherAndReleaseDate({
    name,
    publisher,
    release_date,
  }: ICreateGameDTO): Promise<Game | undefined> {
    log.debug('Games :: findByNameAndPublisherAndReleaseDate');
    return this.ormRepository.findOne({
      where: { name, publisher, release_date: new Date(release_date) },
    });
  }

  public async index(): Promise<Game[]> {
    log.debug('Games :: index');
    return this.ormRepository.find();
  }

  public async findById(id: string): Promise<Game | undefined> {
    log.debug('Games :: findById');
    return this.ormRepository.findOne(id);
  }

  public async create(gameData: ICreateGameDTO): Promise<Game> {
    log.debug('Games :: create');
    const game = this.ormRepository.create(gameData);
    await this.ormRepository.save(game);
    return game;
  }

  public async save(game: Game): Promise<Game> {
    log.debug('Games :: save');
    return this.ormRepository.save(game);
  }
}

export default GamesRepository;

\end{lstlisting}
\vspace{5mm}

\paragrafo{Até aqui, todos os níveis apresentados na Figura \ref{fig:carch} já
foram representados através das classes mencionadas nesta seção, do nível mais
alto da aplicação até a dependência da camada de banco de dados. O mesmo será
feito para a dependência da camada web.} \paragrafo{A classe GamesController tem
uma dependência explícita do módulo CreateGameService. Além disso, também
depende de outros casos de uso da aplicação, como a atualização de dados de
jogos (UpdateGameService) e a listagem ou exibição de detalhes dos mesmos
(IndexGameService e ShowGameService).}

\begin{lstlisting}[language=JavaScript, caption={Adaptador de Interfaces da Web},captionpos=b, label=alg:gamescontroller]    
export default class GamesController {
  public async index(request: Request, response: Response): Promise<Response> {
    const indexGame = container.resolve(IndexGameService);
    const games = await indexGame.execute();

    return response.json(games);
  }

  public async create(request: Request, response: Response): Promise<Response> {
    const { name, price, publisher, release_date } = request.body;
    const createGame = container.resolve(CreateGameService);
    const game = await createGame.execute({
      name,
      price,
      publisher,
      release_date,
    });

    return response.json(classToClass(game));
  }

  public async show(request: Request, response: Response): Promise<Response> {
    const { game_id } = request.params;
    const showGame = container.resolve(ShowGameService);

    const game = await showGame.execute({ game_id });

    return response.json(classToClass(game));
  }

  public async update(request: Request, response: Response): Promise<Response> {
    const { game_id } = request.params;
    const { price, release_date } = request.body;

    const updateGame = container.resolve(UpdateGameService);

    const game = await updateGame.execute({
      game_id,
      price,
      release_date,
    });
  }
}    
\end{lstlisting}

\vspace{5mm}

\paragrafo{Assim como a classe GamesRepository, a GamesController é um tipo de
Adaptador de Interface que se comunica diretamente com o \emph{Express JS},
nosso framework que lida com as requisições HTTP.}
\end{subsecao}
\begin{subsecao}{Os benefícios}\label{subsec:dev_carch_benefits} \paragrafo{O
benefício de adotar esse método de desenvolvimento se torna óbvio ao necessitar
reutilizar o caso de uso em algum outro fluxo da aplicação. Neste caso, supondo
que haja a necessidade de cadastrar grandes volumes de jogos na nossa
plataforma, seria mais conveniente passar toda essa informação através de um
arquivo do que inseri-la manualmente através de requisições Web. Por conta
disso, foi criado uma nova classe, GamesCreateBatch, que processa um arquivo csv
e invoca o serviço de criação de games para cada linha.}
\vspace{5mm}
\begin{lstlisting}[language=JavaScript, caption={Caso de Uso reaproveitado para processamento de arquivos},captionpos=b, label=alg:gamescreatebatch]
export default class GamesCreateBatch implements IBatchOperation {
  public async exec(): Promise<void> {
    //valida o nome do arquivo passado via linha de comando
    //cria lista de linhas

    const createGame = container.resolve(CreateGameService);

    await Promise.all(lines //percorre cada linha da lista criada
      .map(async (line: string): Promise<Game | boolean> => {
        try {
          //validacao da linha

          return await createGame.execute(/* conteudo da linha */));
        } catch (error) {
          return log.error(error);
        }
      }));
  }
}
\end{lstlisting}
\vspace{5mm}

\paragrafo{Como os níveis desse fluxo foram muito bem segregados ao longo dos
módulos, não foi necessário alterar nenhuma linha de código das classes
responsáveis por conhecer as regras de negócios da aplicação e das responsáveis
por interagir com o banco de dados.} \paragrafo{Outro ponto é que se em algum
momento for necessário trocar o \emph{Express JS}, somente a classe
GamesController deverá sofrer uma manutenção, o restante, pelo menos referente a
esse fluxo, continuará intacto.} \paragrafo{Outro ganho da adoção desse método
está na facilidade em testar os componentes dessa aplicação. Por exemplo, o
módulo CreateGameService.spec.ts testa, com simplicidade, 100\% das linhas do
Caso de Uso abordado.}

\vspace{5mm}
\begin{lstlisting}[language=JavaScript, caption={CreateGameService.spec.ts},captionpos=b, label=alg:creategametest]

import CreateGameService from '@modules/games/services/CreateGameService';
import AppError from '@shared/errors/AppError';
import FakeGamesRepository from '../fakes/repositories/FakeGamesRepository';

let fakeGamesRepository: FakeGamesRepository;
let createGame: CreateGameService;

describe('CreateGame', () => {
  beforeEach(() => {
    fakeGamesRepository = new FakeGamesRepository();
    createGame = new CreateGameService(fakeGamesRepository);
  });
  it('should be able to create a new game', async () => {
    const game = await createGame.execute({
      name: 'Cyberpunk',
      price: 80.0,
      publisher: 'CD Projekt Red',
      release_date: new Date(2021, 1, 1),
    });

    expect(game).toHaveProperty('id');
  });
  it('should not be able to create a duplicated game', async () => {
    const releaseDate = new Date(2021, 1, 1);
    await fakeGamesRepository.create({
      name: 'Cyberpunk',
      price: 80.0,
      publisher: 'CD Projekt Red',
      release_date: releaseDate,
    });
    await expect(
      createGame.execute({
        name: 'Cyberpunk',
        price: 80.0,
        publisher: 'CD Projekt Red',
        release_date: releaseDate,
      }),
    ).rejects.toBeInstanceOf(AppError);
  });
});
\end{lstlisting}
\vspace{5mm}

\paragrafo{Para auxiliar nesse propósito, foi criado a classe
FakeGamesRepository, cujo propósito está em simular o acesso ao banco de dados.
Como essa classe é uma implementação da interface IGamesRepository, também não
há necessidade de realizar nenhuma alteração no Caso de Uso nem para a
realização dos testes e nem caso haja a necessidade de trocar o \emph{TypeORM}
por algum outro framework de acesso a banco de dados.}

\vspace{5mm}
\begin{lstlisting}[language=JavaScript, caption={Outra implementação do IGamesRepository},captionpos=b, label=alg:fakegamesrepository]
export default class FakeGamesRepository implements IGamesRepository {
  private games: Game[] = [];

  public async findByNameAndPublisherAndReleaseDate({
    name,
    publisher,
    release_date,
  }: ICreateGameDTO): Promise<Game | undefined> {
    return this.games.find(
      game => game.name === name && game.publisher === publisher,
    );
  }

  public async create(gameData: ICreateGameDTO): Promise<Game> {
    const game = new Game();
    Object.assign(game, { id: uuidv4() }, gameData);
    this.games.push(game);
    return game;
  }
\end{lstlisting}
\vspace{5mm}

\end{subsecao}
\end{secao}
\begin{secao}{Desenvolvendo uma aplicação SOLID} \label{sec:dev_solid}

\paragrafo{No fluxo descrito na seção anterior já é possível encontrar alguns
conceitos SOLID sendo implementados na aplicação web. Porém, será necessário ver
outros fluxos para obter exemplos de implementação que sigam todos os
princípios.}

\begin{subsecao}{A implementação}\label{subsec:dev_solid_implementation}

\vspace{5mm}
\begin{lstlisting}[language=JavaScript, caption={ShowGameService.ts},captionpos=b, label=alg:showgameservice]
import AppError from '@shared/errors/AppError';
import log from '@shared/utils/log';
import { injectable, inject } from 'tsyringe';
import Game from '../infra/typeorm/entities/Game';

import IGamesRepository from '../repositories/IGamesRepository';

interface IRequest {
  game_id: string;
}

@injectable()
export default class ShowGameService {
  constructor(
    @inject('GamesRepository')
    private gamesRepository: IGamesRepository,
  ) {}

  public async execute({ game_id }: IRequest): Promise<Game> {
    log.debug(`Show Game :: id: ${game_id}`);
    const game = await this.gamesRepository.findById(game_id);
    if (!game) {
      throw new AppError('Game not found', 404);
    }
    return game;
  }
}
\end{lstlisting}
\vspace{5mm}
  
\paragrafo{O Princípio da Responsabilidade Única (\acs{SRP}) está sendo seguido
na classe ShowGameService. Isso ocorre pois a classe tem apenas um propósito,
que é a exibição dos detalhes do jogo. Sendo assim, a mesma tem um escopo mais
definido em razões para necessitar de alteração: Adicionar validações, passar a
buscar os jogos pelo nome em vez do id, entre outros.} \paragrafo{Uma forma
fácil de imaginar a classe acima ferindo este princípio seria se a mesma além de
exibir jogos, também criasse ou atualizasse os mesmos. Com isso, a manutenção
dessa classe pode ser mais difícil de se realizar do que a da primeira:}

\vspace{5mm}
\begin{lstlisting}[language=JavaScript, caption={Implementação que viola o SRP},captionpos=b, label=alg:gameservice]
@injectable()
export default class GameService {
  constructor(
    @inject('GamesRepository')
    private gamesRepository: IGamesRepository,
  ) {}

  public async showGame({ game_id }: IShowGameDTO): Promise<Game> {
    //validacoes e processos para exibir detalhes de um jogo
  }

  public async createGame({
      name,
      price,
      publisher,
      release_date,
    }: ICreateGameDTO): Promise<Game> {

    //validacoes e processos para criar um jogo
  }

  public async updateGame({
    game_id,
    price,
    release_date,
  }: IUpdateGameDTO): Promise<Game> {
    const game = await this.gamesRepository.findById(game_id);

    //validacoes e processos para atualizar um jogo
  }
}
\end{lstlisting}
\vspace{5mm}

\paragrafo{Desta forma, o leitor pode se questionar porque a classe
 GamesController não viola o \acs{SRP}, visto que a mesma realiza todas essas
 operações. Isso não se caracteriza como uma violação pois a responsabilidade da
 classe é apenas comunicar as chamadas recebidas via HTTP para a aplicação. O
 \emph{Controller} desconhece os fluxos de exibição, criação ou atualização, mas
 conhece as classes que contêm esses fluxos.}

\paragrafo{Para exemplificar o Princípio Aberto Fechado (\acs{OCP}), existe um
fluxo na aplicação responsável pela reinvidicação de cartões pré-pagos. Para
usar esse serviço, o usuário insere o código do cartão e resgata jogos ou
dinheiro para a compra de jogos na loja.} \paragrafo{Porém, ao cadastrar esses
códigos no sistema, a validação para esses dois tipos de items a ser
reinvidicados é diferente. Os cartões de dinheiro devem resgatar somente os
valores de R\$30, R\$50 ou R\$100. Para validar um jogo, somente é necessário
que este exista.}
\paragrafo{Com os requisitos acima, pode-se pensar em criar a seguinte classe:}

\vspace{5mm}
\begin{lstlisting}[language=JavaScript, caption={CreateCodeService.ts},captionpos=b, label=alg:createcodeservice]
import { injectable, inject } from 'tsyringe';
import { v4 as uuidv4 } from 'uuid';

import AppError from '@shared/errors/AppError';
import IGamesRepository from '@modules/games/repositories/IGamesRepository';
import IGamestoreCodesRepository from '../repositories/IGamestoreCodeRepository';
import GamestoreCode from '../infra/typeorm/entities/GamestoreCode';

interface IRequest {
  cash?: number;
  game_id?: string;
}

@injectable()
export default class CreateCodeService {
  constructor(
    @inject('GamestoreCodesRepository')
    private gamestoreCodesRepository: IGamestoreCodesRepository,

    @inject('GamesRepository')
    private gamesRepository: IGamesRepository,
  ) {}

  public async execute(request: IRequest): Promise<GamestoreCode | undefined> {
    let code;
    const { cash, game_id } = request;

    if (!cash && !game_id) {
      throw new AppError('Internal Server Error', 500);
    }

    if (cash) {
      if (cash !== 30 && cash !== 50 && cash !== 100) {
        throw new AppError('Invalid cash quantity');
      }

      code = await this.gamestoreCodesRepository.create({
        code: uuidv4(),
        is_redeemed: false,
        product: {
          cash,
        },
      } as GamestoreCode);
    }

    if (game_id) {
      if (!(await this.gamesRepository.findById(game_id))) {
        throw new AppError('Game does not exist');
      }

      code = await this.gamestoreCodesRepository.create({
        code: uuidv4(),
        is_redeemed: false,
        product: {
          game: game_id,
        },
      } as GamestoreCode);
    }
    return code;
  }
}

\end{lstlisting}
\vspace{5mm}

\paragrafo{Porém, essa estratégia fere o \acs{OCP}. Note que, se em algum
momento no futuro for necessário reinvidicar um terceiro item, teremos que
adicionar mais um if ao método principal dessa classe, o que deixará o código
ainda mais acoplado.} \paragrafo{Para ilustrar isso, suponhamos que será
necessário reinvidicar códigos que resgatem Pacotes de Expansão para jogos já
cadastrados em nossa plataforma. Essa funcionalidade extrapola o escopo desse
trabalho, mas sua implementação nessa classe seria algo como o exemplo a
seguir:}

\vspace{5mm}
\begin{lstlisting}[language=JavaScript, caption={Implementação que viola o OCP},captionpos=b, label=alg:ocpviolation]
interface IRequest {
  cash?: number;
  game_id?: string;
  dlc_id?: string;
}

@injectable()
export default class CreateCodeService {
  constructor(
    @inject('GamestoreCodesRepository')
    private gamestoreCodesRepository: IGamestoreCodesRepository,

    @inject('GamesRepository')
    private gamesRepository: IGamesRepository,

    @inject('DlcsRepository')
    private dlcsRepository: IDlcsRepository,
  ) {}

  public async execute(request: IRequest): Promise<GamestoreCode | undefined> {
    if (!request.cash && !request.game_id && !request.dlc_id) {
      throw new AppError('Internal Server Error', 500);
    }

    if (request.cash) {
      //validacao de cash (R$30, R$50 ou R$100)
      //cadastro do cartao pre pago de cash
    }

    if (request.game_id) {
      //validacao de game (se o game existe)
      //cadastro do cartao pre pago de gane
    }

    if (request.dlc_id) {
      //validacao de dlc (se a dlc existe)
      //cadastro do cartao pre pago de gane
    }
  }
}
\end{lstlisting}
\vspace{5mm}

\paragrafo{Para não ferir este princípio, basta estendermos a funcionalidade de
uma classe base e realizar as validações nas classes herdeiras.} \paragrafo{A
seguir, temos a mesma funcionalidade, mas respeitando o \acs{OCP}:}

\vspace{5mm}
\begin{lstlisting}[language=JavaScript, caption={Implementação de acordo com o OCP},captionpos=b, label=alg:ocpexample]  
//O modulo CreateDlcCodeService.ts
import { injectable, inject } from 'tsyringe';
import { v4 as uuidv4, validate } from 'uuid';

import AppError from '@shared/errors/AppError';
import IDlcsRepository from '@modules/games/repositories/IDlcsRepository';
import IGamestoreCodesRepository from '../repositories/IGamestoreCodeRepository';
import GamestoreCode from '../infra/typeorm/entities/GamestoreCode';
import AbstractCodeTemplate from './AbstractCodeTemplate';

interface IRequest {
  dlc_id: string;
}

@injectable()
export default class CreateGameCodeService extends AbstractCodeTemplate {
  constructor(
    @inject('DlcsRepository')
    private dlcsRepository: IDlcsRepository,

    @inject('GamestoreCodesRepository')
    private gamestoreCodesRepository: IGamestoreCodesRepository,
  ) {
    super();
  }

  protected async createCode({ dlc_id }: IRequest): Promise<GamestoreCode> {
    const code = await this.gamestoreCodesRepository.create({
      code: uuidv4(),
      is_redeemed: false,
      product: {
        game: dlc_id,
      },
    } as GamestoreCode);
    return code;
  }

  protected async validateData({ dlc_id }: IRequest): Promise<void> {
    if (!validate(dlc_id) || !(await this.dlcsRepository.findById(dlc_id))) {
      throw new AppError('DLC does not exist');
    }
  }
}
  \end{lstlisting}
\vspace{5mm}

\paragrafo{Um exemplo muito recorrente dos dois pŕoximos princípios, isto é, o
Princípio da Substituição de Liskov (\acs{LSP}) e o Princípio da Segregação de
Interfaces (\acs{ISP}), está no funcionamento das classes \emph{Repository} da
aplicação.} \paragrafo{Tomando por exemplo o \emph{Repository} de Usuários,
temos a interface IUsersRepository e suas implementações UsersRepository e
FakeUsersRepository.}

\vspace{5mm}
\begin{lstlisting}[language=JavaScript, caption={Interace a ser implementada},captionpos=b, label=alg:iusersrepository]
import ICreateUserDTO from '../dtos/ICreateUserDTO';
import User from '../infra/typeorm/entities/User';

export default interface IUsersRepository {
  findById(id: string): Promise<User | undefined>;
  findByEmail(email: string): Promise<User | undefined>;
  create(data: ICreateUserDTO): Promise<User>;
  save(user: User): Promise<User>;
}
\end{lstlisting}
\vspace{5mm}

\vspace{5mm}
\begin{lstlisting}[language=JavaScript, caption={UsersRepository.ts},captionpos=b, label=alg:usersrepository]
import { getRepository, Repository } from 'typeorm';
import IUsersRepository from '@modules/users/repositories/IUsersRepository';

import ICreateUserDTO from '@modules/users/dtos/ICreateUserDTO';
import log from '@shared/utils/log';
import User from '../entities/User';

class UsersRepository implements IUsersRepository {
  private ormRepository: Repository<User>;

  constructor() {
    this.ormRepository = getRepository(User);
  }

  public async findById(id: string): Promise<User | undefined> {
    log.debug('Users :: findById');
    return this.ormRepository.findOne(id);
  }

  public async findByEmail(email: string): Promise<User | undefined> {
    log.debug('Users :: findByEmail');
    const user = await this.ormRepository.findOne({
      where: { email },
    });
    return user;
  }

  public async create(userData: ICreateUserDTO): Promise<User> {
    log.debug('Users :: create');
    const user = this.ormRepository.create(userData);
    await this.ormRepository.save(user);
    return user;
  }

  public async save(user: User): Promise<User> {
    log.debug('Users :: save');
    return this.ormRepository.save(user);
  }
}

export default UsersRepository;
\end{lstlisting}
\vspace{5mm}

\vspace{5mm}
\begin{lstlisting}[language=JavaScript, caption={FakeUsersRepository.ts},captionpos=b, label=alg:fakeusersrepository]
import IUsersRepository from '@modules/users/repositories/IUsersRepository';
import ICreateUserDTO from '@modules/users/dtos/ICreateUserDTO';

import { v4 as uuidv4 } from 'uuid';
import User from '../../../../modules/users/infra/typeorm/entities/User';

export default class FakeUsersRepository implements IUsersRepository {
  private users: User[] = [];

  public async findById(id: string): Promise<User | undefined> {
    return this.users.find(user => user.id === id);
  }

  public async findByEmail(email: string): Promise<User | undefined> {
    return this.users.find(user => user.email === email);
  }

  public async create(userData: ICreateUserDTO): Promise<User> {
    const user = new User();
    Object.assign(user, { id: uuidv4() }, userData);
    this.users.push(user);
    return user;
  }

  public async save(user: User): Promise<User> {
    const findIndex = this.users.findIndex(findUser => findUser.id === user.id);
    this.users[findIndex] = user;
    return user;
  }
}
\end{lstlisting}
\vspace{5mm}

\paragrafo{Podemos dizer que o \acs{LSP} é respeitado na relação entre esses
três módulos pois qualquer uma das implementações da interface IUsersRepository
é válida de ser importada no módulo CreateUserService, visto que nenhuma delas
quebraria a aplicação.} \paragrafo{O mesmo não seria verdade se, por exemplo, a
interface não especificasse um método de criação de usuário. Na hora que esse
método fosse invocado no Caso de Uso, a aplicação retornaria um erro.}

\vspace{5mm}
\begin{lstlisting}[language=JavaScript, caption={Violando o LSP},captionpos=b, label=alg:lspviolation]

  // Modulo IUsersRepository
  export default interface IUsersRepository {
    findById(id: string): Promise<User | undefined>;
    findByEmail(email: string): Promise<User | undefined>;
    save(user: User): Promise<User>;
    // metodo create removido
  }

  // Modulo UsersRepository
  class UsersRepository implements IUsersRepository {
  private ormRepository: Repository<User>;

  constructor() {
    this.ormRepository = getRepository(User);
  }

  public async create(userData: ICreateUserDTO): Promise<User> {
    const user = this.ormRepository.create(userData);
    await this.ormRepository.save(user);
    return user;
  }

  //implementacao dos outros metodos da interface
}

//Modulo CreateUserService
export default class CreateUserService {
  constructor(
    @inject('UsersRepository')
    private usersRepository: IUsersRepository,

    @inject('HashProvider')
    private hashProvider: IHashProvider,
  ) {}

  public async execute(data: IRequest): Promise<User> {
    // validacao dos dados

    // linha abaixo tem erro
    const user = await this.usersRepository.create(/* dados para criacao do usuario */);

    await this.usersRepository.save(user);

    return user;
  }
}

  \end{lstlisting}
\vspace{5mm}

\notaeditor{Nota: Completar a seção com o ISP e DIP}

\end{subsecao}
\end{secao}
\end{capitulo}